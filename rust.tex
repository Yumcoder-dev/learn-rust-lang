\documentclass{beamer}
%\usepackage[utf8]{inputenc}
%\usepackage{graphicx}
%usepackage {mathtools}
%\usepackage{utopia} %font utopia imported
\usetheme{CambridgeUS}
%\usecolortheme{default}
\usepackage{minted}

% set colors
\definecolor{greyColor}{RGB}{242,242,245} % grey
\definecolor{greenColor}{RGB}{51,112,98} % green
\definecolor{yellowColor}{RGB}{246,202,84} % yellow

\setbeamercolor*{palette primary}{bg=yellowColor}
\setbeamercolor*{palette secondary}{bg=greyColor}
\setbeamercolor*{palette tertiary}{bg=greenColor, fg = white}
\setbeamercolor*{titlelike}{fg=greyColor}
\setbeamercolor*{title}{bg=greenColor, fg = white}
\setbeamercolor*{item}{fg=greenColor}
\setbeamercolor*{caption}{fg=greenColor}

\setbeamercolor{frametitle}{fg=black,bg=greyColor}

%\usefonttheme{professionalfonts}
%\usepackage{natbib}
%\usepackage{hyperref}
%------------------------------------------------------------

\title[Rust-Lang]{Rust Programming Language}
\author[Yumcoder]{Yumcoder}

\institute[UoT]{University of Toronto}
\date[{\today} ]
{\today}

%------------------------------------------------------------
%This block of commands puts the table of contents at the 
%beginning of each section and highlights the current section:
%\AtBeginSection[]
%{
%  \begin{frame}
%    \frametitle{Contents}
%    \tableofcontents[currentsection]
%  \end{frame}
%}
\AtBeginSection[]{
  \begin{frame}
  \vfill
  \centering
  \begin{beamercolorbox}[sep=8pt,center,shadow=true,rounded=true]{title}
    \usebeamerfont{title}\insertsectionhead\par%
  \end{beamercolorbox}
  \vfill
  \end{frame}
}
%------------------------------------------------------------

\begin{document}

%The next statement creates the title page.
\frame{\titlepage}
\begin{frame}
\frametitle{Contents}
\tableofcontents
\end{frame}
%------------------------------------------------------------
\section{Installation}
    \begin{frame}{Installing rustup on Linux or macOS}
    	test 2
    	One-line code formatting also works with \texttt{minted}. For example, a small fragment of HTML like this:
    	\mint{html}|<h2>Something <b>here</b></h2>|
    	\noindent can be formatted correctly.
    \end{frame}

\begin{frame}[fragile]
	\frametitle{Python Code listing in Beamer}
	The following Python code adds two numbers and display the result using \verb|print()| function:
	%\rule{\textwidth}{1pt}
	\scriptsize
	\begin{minted}[bgcolor=greyColor,linenos, breaklines,frame=leftline, numbersep=1pt,mathescape]{python}
# This program adds two numbers
num1 = 1.5
num2 = 6.3
# Add two numbers
sum = num1 + num2
# Display the sum
print('The sum of {0} and {1} is {2}'.format(num1, num2, sum))
	\end{minted}
	%\rule{\textwidth}{1pt}
\end{frame}

\section{Summary}
    \begin{frame}{Summary}
 Summary of the above...
    \end{frame}

\section*{Acknowledgement}  
\begin{frame}
\Huge{\centerline{Thank you!}}
\end{frame}

\end{document}



