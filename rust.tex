\documentclass{beamer}
%\usepackage[utf8]{inputenc}
%\usepackage{graphicx}
%usepackage {mathtools}
%\usepackage{utopia} %font utopia imported
\usetheme{CambridgeUS}
%\usecolortheme{default}
\usepackage{minted}
%\usefonttheme{professionalfonts}
%\usepackage{natbib}
%\usepackage{hyperref}
%------------------------------------------------------------
% set colors
\definecolor{greyColor}{RGB}{242,242,245} % grey
\definecolor{greenColor}{RGB}{51,112,98} % green
\definecolor{yellowColor}{RGB}{246,202,84} % yellow
%------------------------------------------------------------
\setbeamercolor*{palette primary}{bg=yellowColor}
\setbeamercolor*{palette secondary}{bg=greyColor}
\setbeamercolor*{palette tertiary}{bg=greenColor, fg = white}
\setbeamercolor*{titlelike}{fg=greyColor}
\setbeamercolor*{title}{bg=greenColor, fg = white}
\setbeamercolor*{item}{fg=greenColor}
\setbeamercolor*{caption}{fg=greenColor}
\setbeamercolor{frametitle}{fg=black,bg=greyColor}
% see https://pygments.org/styles/
\usemintedstyle{friendly}
% [bgcolor=greyColor,linenos, breaklines,frame=leftline, numbersep=1pt,mathescape]
\setminted[shell]{bgcolor=greyColor, breaklines, numbersep=1pt,mathescape}
\setminted[rust]{bgcolor=greyColor,linenos, breaklines,frame=leftline, numbersep=1pt,mathescape}
%------------------------------------------------------------
\title[Rust-Lang]{Rust Programming Language}
\author[Yumcoder]{Yumcoder}

\institute[UoT]{University of Toronto}
\date[{\today} ]
{\today}
%------------------------------------------------------------
%This block of commands puts the table of contents at the 
%beginning of each section and highlights the current section:
%\AtBeginSection[]
%{
	%  \begin{frame}
		%    \frametitle{Contents}
		%    \tableofcontents[currentsection]
		%  \end{frame}
	%}
\AtBeginSection[]{
	\begin{frame}
		\vfill
		\centering
		\begin{beamercolorbox}[sep=8pt,center,shadow=true,rounded=true]{title}
			\usebeamerfont{title}\insertsectionhead\par%
		\end{beamercolorbox}
		\vfill
	\end{frame}
}
%------------------------------------------------------------

\begin{document}
	
	%The next statement creates the title page.
	\frame{\titlepage}
	\begin{frame}
		\frametitle{Contents}
		\tableofcontents
	\end{frame}
	%------------------------------------------------------------
	\section{Installation}
	\begin{frame}[fragile]
		\frametitle{Installing rustup on Linux or macOS}
		If you’re using Linux or macOS, open a terminal and enter the following command:
		\scriptsize
		\inputminted{shell}{./code/install.shell}
		If the install is successful, the following line will appear:
		\begin{minted}[linenos=false, breaklines,frame=none]{shell}
			Rust is installed now. Great!
		\end{minted}
	To check whether you have Rust installed correctly, open a shell and enter this line:
	\inputminted{shell}{./code/install-check.shell}
	\end{frame}
	
	\begin{frame}[fragile]
		\frametitle{Updating, Uninstalling and Local Documentation}
		Once Rust is installed via rustup, when a new version of Rust is released, updating to the latest version is easy. From your shell, run the following update script:
		\scriptsize
		\inputminted{shell}{./code/install-update.shell}
		
		To uninstall Rust and rustup, run the following uninstall script from your shell:
		\inputminted{shell}{./code/install-uninstall.shell}
		
		The installation of Rust also includes a local copy of the documentation, so you can read it offline. Run \mintinline{shell}{rustup doc}  to open the local documentation in your browser.
	\end{frame}
	
	\begin{frame}[fragile]
		\frametitle{Hello, World!}
		Open a terminal and enter the following commands to make a projects directory and a directory for the “Hello, world!” project within the projects directory.
		\scriptsize
		\inputminted[linenos, breaklines,frame=leftline, numbersep=1pt]{shell}{./code/hello-world.shell}
		
		Filename: main.rs
		\inputminted{rust}{./code/hello-world-main.rs}
	\end{frame}

	\begin{frame}[fragile]
		\frametitle{title}
	\end{frame}
	
	\section{Summary}
	
	\begin{frame}{Summary}
		Summary of the above...
	\end{frame}
	
	\section*{Acknowledgement}  
	\begin{frame}
		\Huge{\centerline{Thank you!}}
	\end{frame}
	
\end{document}

